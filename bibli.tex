
\cleardoublepage                         % 保證奇數頁為章節起始
\phantomsection
\addcontentsline{toc}{chapter}{索引}     % 將索引加入目錄中

\printindex

\cleardoublepage                         % 保證奇數頁為章節起始
\phantomsection
\addcontentsline{toc}{chapter}{參考文獻}     % 將參考文獻加入目錄中

\bibliographystyle{unsrt}      
%\bibliography{myfoo}                   % 或標準資料庫用法          
\begin{thebibliography}{20}          % 文獻少寫法
\bibitem{knu84}
Donald E. Knuth. \emph{{The TEXbook, Volume A of Computers and Typesetting}}. \hskip 1em plus 0.5em minus
 0.4em\relax Addison-Wesley, Reading, Massachusetts, second edition, 1984, ISBN 0-201-13448-9.\\
\url{http://www-cs-staff.stanford.edu/~knuth/index.html}

\bibitem{lam94}
Leslie Lamport. \emph{{\LaTeX{}: A Document Preparation System}}. \hskip 1em plus 0.5em minus 0.4em\relax Addison-Wesley, Reading, Massachusetts, second edition, 1994, ISBN 0-201-52983-1.

\bibitem{lo12a}
J.~LO, \emph{{eThinking in Circuits with PSpice}}.\hskip 1em plus
 0.5em minus 0.4em\relax Cavesbooks, Inc., 2012, \\ ISBN 978-957-41-8721-8.

\bibitem{lo12b}
------, \emph{{aThinking in Control with Matlab}}.\hskip 1em plus
0.5em minus 0.4em\relax Cavesbooks, Inc., 2012, \\ ISBN pending.

\bibitem{lo12c}
------, \emph{\LaTeX\ \& U 自助出版}.\hskip 1em plus
0.5em minus 0.4em\relax 中央敦煌, 北科文具部, 2012, \\ ISBN 978-957-41-9448-3.

\bibitem{lo12d}
------, \emph{Packages author of ncuthesis(CJK, Xe), bizcard, cnwritingCJK}.\hskip 1em plus
0.5em minus 0.4em\relax Free packages, 2012.\\
\url{https://code.google.com/p/ncu-thesis-latex-template/}
\bibitem{}
\emph{{Writing a thesis in \LaTeX}}
\url{http://texblog.org/}

\bibitem{126570}
Chinese character \textbackslash cjk within \textbackslash section\{\} does not work using pdflatex, + \textbackslash includegraphics, \hskip 1em plus
0.5em minus 0.4em\relax\\
\url{http://tex.stackexchange.com/a/126570}

\bibitem{79776}
\emph{Page numbers only appear on pages where a chapter starts}, \hskip 1em plus
0.5em minus 0.4em\relax\\
\url{http://tex.stackexchange.com/a/79776}
\end{thebibliography}                      % 少文獻寫法

