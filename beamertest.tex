\documentclass[a4paper,12pt]{beamer}
\usepackage{fontspec}
\usepackage[CJKnumber]{xeCJK}
\usepackage{xunicode}           % unicode character macros
\setmainfont{Times New Roman}
\setCJKmainfont{標楷體} %kaiu.ttf 
\XeTeXlinebreaklocale "zh"
\XeTeXlinebreakskip = 0pt plus 1pt
\usetheme[secheader]{Boadilla}
%其它需求
\begin{document}
\title{主標題}
\subtitle{副標題}
\author{作者}
\institute{單位}
\date{\today}
\maketitle
%----------------------------
\section{Section}
\begin{frame}[t]
\frametitle{重點整理}
可用\textbackslash{section},
\textbackslash{subsection}於frame外。
然後在這一frame內用指令\textbackslash tableofcontents
,則所有節、小節名皆會收錄於此,產生目錄,如下所示。
\tableofcontents
\end{frame}
%----------------------------
\subsection{Subsection}
\begin{frame}[t]
\frametitle{第一張}
依此類推,將論文精簡直接剪貼,放入frame內。
若需以節(section)或小節(subsection)來分割內容則於適當
處(frame外)加\textbackslash section\{...\},
\textbackslash subsection\{...\}。
更複雜的結構。上網學習了。
\vspace{1cm}

{\Large 如果這論文手冊用心,
請告訴其他研究生,讓他們的論文快速、
排版精美、一生回憶。}
\end{frame}
%----------------------------
\end{document}