\chapter{\protect 語法入門}
首先,每章結構如下所示。

\begin{Verbatim}[frame=single,firstline=1,label=Every chapter]
\chapter{章名}           % 宣告某章開始
...
\section{節名一}         % 宣告某節開始
...
\section{節名二}
...
\subsection{小節名}       % 宣告小節開始
...
\subsubsection{小小節名}         % 宣告小小節開始
...
\end{Verbatim}
\index{Reserved words}
\begin{itemize}
\item 所有\LaTeX指令都由反斜線開頭,例如\textbackslash command (不可有{\tt arabic}數字)。
\item \LaTeX\ 保留字有十個: \textbackslash\  \$\ \{\ \}\ $\tilde{}$\ \#\ \%\ \&\ \textasciicircum\
 \_\ $>$, $<$ 不可單獨使用。  
\item \% 代表註解,其後任何字或空白皆忽略。
\item \LaTeX\ 一般檔案的副檔名為{\tt filename.tex},文獻檔則為\newline {\tt filename.bib}。其他都是編譯時產生輔助檔,可刪除。
\item 一個空白與連續數行的空白\LaTeX{}都是為一個空白,故主檔寫作時可適當留白,方便作者自行閱讀。
\item 環境名\textbackslash begin \ldots \textbackslash end,任何括號 $\{_1$ $[_2$ $(_{3}$ \ldots $)_3$ $]_2$ $\}_1$ 皆要對稱,不可錯置。
\end{itemize}
用{\textbackslash chapter\{章名\}}指令後,告訴\LaTeX{}以下文字要自成一章,就像打字一般,直接輸入中英文。
這樣就可以完成大部分的論文主體了。好用吧!!! 論文裡還有什麼
要寫?數學式啦!這是\LaTeX\ 的強項,我們稍後再說。先繼續介紹章節用法,這裡有假設、求解、驗證、數理基礎 四個小節({\tt section})。數理基礎又分次小節({\tt subsection})內含論文撰寫常用的技巧。

\section{假設}
若需要分章節則用{\textbackslash section\{節名\}},再繼續打字。
\section{求解}
同樣概念{\textbackslash section\{節名\}},繼續下去。若要換新的一章則開新檔{\tt chapter3.tex},內部第一行又如前所述

\section{驗證}
寫完後存檔。再以pdf\LaTeX\ 或 Xe\LaTeX\ 編譯。則可看到輸出。當然這一切都需在\TeX/\LaTeX環境之下。這方面知識網路很多。可自行上網學習。入門技巧約需1小時,主要是數學式寫法,現在我們將學一些初步技巧。
\section{數理基礎}
\index{\LaTeX!\textbackslash chapter}
\index{\LaTeX!\textbackslash section}
\index{\LaTeX!\textbackslash subsection} \index{\LaTeX\ 環境!Verbatim}%
數學式寫法很簡單,一行文字中含數學式是這樣寫\verb|$A^1_{234}$|,會產生文字中的數學式$A^1_{234}$,此例說明數學上下標的寫法,也說明有四個保留字不能亂用,因為保留給數學式\$\ 的上下標及群組概念了。數學式只能出現在文字內({\tt inline math mode})像這樣$1+\frac{1}{1+\frac{2}{1+\frac{3}{4}}}$,也不夠美觀。要單獨成一數學環境則用\\
\begin{Verbatim}[frame=single,firstline=1,label=Form 1 w/o number]
$$ 或 \[
1+\frac{1}{1+\frac{2}{1+\frac{3}{4}}}, \\
\frac{-b\pm\sqrt{b^2-4ac}}{2a}, \\
\quad \frac{a_1^{3x}+a_2^{-3x}}{a_1^x+a_2^{-x}}
$$ 或 \]
\end{Verbatim}
會單獨成一漂亮的數學式如下  \index{\LaTeX\ 環境!\textbackslash [}\index{\LaTeX\ 環境!\textbackslash ]}
$$ 
1+\frac{1}{1+\frac{2}{1+\frac{3}{4}}}, \\
\frac{-b\pm\sqrt{b^2-4ac}}{2a},\\
\quad \frac{a_1^{3x}+a_2^{-3x}}{a_1^x+a_2^{-x}}
$$
這樣就漂亮多了。注意在此數學環境強置換行(\textbackslash \textbackslash)無效,因結果是同一行印出,這一例題也說明
\begin{itemize}
\item 數學式有對齊的需要(聯立方程式)。
\item 適當留白亦有美觀效果,注意數學式間間距不同?如何達成\footnote{\textbackslash qquad,\textbackslash quad,\textbackslash ,\textbackslash,,\textbackslash;皆可,只是間距有別。}? 
\item 另外,但這樣寫是無編號的寫法,因為有時候我們需引用(\verb|\ref|)某方程式時則必須用編號的方式。
\end{itemize}
\index{\LaTeX\ 環境!\$\$}
\subsection{方程式}
最簡單的是用{\tt equation}環境,產生下列常微分方程式\\
\begin{Verbatim}[frame=single,firstline=1,label=Form 2 with number]  
\begin{equation}
\mbox{常微分方程式} \quad a \ddot y+ b\dot y +c=f
\label{eqn1}
\end{equation}
\end{Verbatim}
會產生 \index{\LaTeX\ 環境!minipage}
\begin{equation}
\mbox{常微分方程式} \quad a \ddot y+ b\dot y +c=f \index{\LaTeX!\textbackslash mbox}
\label{eqn1}
\end{equation}
文字環境有數學式用{\color{red}\$\mbox{math mode}\$}的方式表現。但數學環境中有文字時,用{\color{red}\verb+\mbox{text mode}+}表現。如上所示\footnote{其實\LaTeX\ 就是文字,數學,物件三大觀念,亦可混合一起使用。}。
而\verb|\label{eqn1}|是為了可稍後用{\verb|\ref{eqn1}|}參照。希望編號及間距的問題解決了,但希望對齊呢?要對齊則用{\tt eqnarray}環境及 \&\& (又一保留字)\\
\begin{Verbatim}[frame=single,firstline=1,label=Form 3 with alighment and number]
\begin{eqnarray}
f &=& a \ddot y+ b\dot y +c \nonumber \\
g &=& a\frac{\partial x}{\partial t_1}
+ b\frac{\partial x}{\partial t_2} +c
\end{eqnarray}
\end{Verbatim}
如上所示,其結果如下\footnote{強制換行在此環境有效。}。                \index{\LaTeX!\textbackslash ref}
\begin{eqnarray}
f&=& a \ddot y+ b\dot y +c \nonumber \\  \index{\LaTeX!\textbackslash nonumber}
g&=& a \frac{\partial x}{\partial t_1}
+ b \frac{\partial x}{\partial t_2}+c        % 最後一行不需 \\
\end{eqnarray}
結果是自動編號的對齊方式,有了數學式編號,但全篇都有用編號也不好,若不須編號則用\verb+\nonumber+加於該數式尾部\footnote{如果只要對齊,都不要編號則用\textbackslash begin\{eqnarray*\}/\textbackslash end\{eqnarray*\}。}
\index{\LaTeX\ 環境!eqnarray}

\subsection{矩陣}
矩陣則用{\tt equation}及{\tt array}環境,可印出如下所示。\index{\LaTeX\ 環境!equation}\\ 
\begin{Verbatim}[frame=single,firstline=1,label=Every matrix]
\begin{equation}
\left [              % 左 [
\begin{array}{ccc}   % c=置中,l=置左,r=置右。
a & b & 1\\
c & d & 2\\
4 & 5 & 6\
\end{array}
\right ]             % 右 ]
\label{mat1}
\end{equation}
\end{Verbatim}
會產生\footnote{不要編號的矩陣,怎麼寫? \qquad {\tt Ans:} \textbackslash [ \textbackslash ] or \$\$ \$\$。} \index{\LaTeX\ 環境!minipage}
\begin{equation}
\left [
\begin{array}{ccc}  \index{\LaTeX\ 環境!array}
a & b & 1\\
c & d & 2\\
4 & 5 & 6
\end{array}
\right ]
\label{mat1} \index{\LaTeX!\textbackslash label}\index{\LaTeX!\textbackslash ref}
\end{equation}
要引述前述之常微分方程式時則用 \verb|\ref{eqn1}|可把在前一頁的該方程式編號(\ref{eqn1})寫出,方便讀者理解所指方程式為何。同理,上一個矩陣亦可用\verb|\ref{mat1}|參照(\ref{mat1})。

下一個例題則是強調使用括號指令時需注意對稱的用法,\verb|\left \{|\footnote{注意:\{\}是保留字,要用它必須這樣用\textbackslash \{ \textbackslash \}} \ldots
\verb|\right .|。從結果可看出,用左大括號後並不需用右大括號,如何處理?如第8行所示。
\index{\LaTeX!\textbackslash footnote}

\begin{Verbatim}[frame=single,firstline=1,label=Pairs]
\begin{equation}
y=\frac{1}{x} \quad  
\left \{            % 左 {
\begin{array}{ll}   % c=置中,l=置左,r=置右
\frac{1}{x} & x \neq 0 \\  
\infty      & x = 0
\end{array}
\right .            % 右 } 不需要 則用此技巧
\end{equation}
\end{Verbatim}
\begin{equation}
y=\frac{1}{x} \quad  \index{\LaTeX!\textbackslash quad}
\left \{
\begin{array}{ll}   
\frac{1}{x} & x \neq 0 \\
\infty      & x = 0  \index{\LaTeX!\textbackslash neq}
\end{array}
\right .            
\end{equation}
\subsection{其他}
還有其他的數學式,例如\\
\begin{Verbatim}[frame=single,firstline=1,label=Various math forms]
\[
\sum_{i=1}^n A_i, \int_0^t f(\tau)\,d\tau, \sqrt{x}, \tan, 
\sin, \pi, \omega, f', \lim_{t\rightarrow \infty} f(t),
\forall x in \Re, \exists, \angle \theta, \bar A, \vec A
\]
\end{Verbatim}
可產生 \index{\LaTeX!\textbackslash sum} \index{\LaTeX!\textbackslash int} \index{\LaTeX!\textbackslash sqrt}
\[
\sum_{i=1}^n A_i,  \int_0^t f(\tau)\,d\tau, \sqrt{x}, \tan, \sin, \pi, \omega, f',
\lim_{t\rightarrow \infty} f(t),
\forall x \in \Re, \exists, \angle \theta, \bar A, \vec A
\]
以上是寫數學的技巧,因頁面限制無法全寫出來,為此,檔案夾內準備了一個網路上搜索來的檔案
\href{./Symbols.pdf}{ Symbols.pdf }內含所有數學式的寫法,方便各位寫數學式時參考。

\section{物件}
這裡是指數學定理,表格,小頁,列舉,插圖等獨立單元。
\index{\LaTeX!\textbackslash newtheorem}\index{\LaTeX!\textbackslash newcommand}

\subsection{定理}
主檔在宣告區有定義{\tt newtheorem}(用來定義使用者定理環境)中文化定理及證明。故可連續使用且連續依章節自動編號。這是\LaTeX\ 的核心觀念 ---文學編程。例如這樣以{\tt newcommand}(用來定義使用者指令)定義中央大學

\begin{Verbatim}[frame=single,firstline=1,label=Simple macro without parameters]
\newcommand{\ncu}{{\color{red} \bf \Huge 中央大學}}
\end{Verbatim}

\newcommand{\ncu}{{\color{red} \bf \Huge 中央大學}}
每次使用\verb|\ncu|則產生紅色、粗體、極大的\ncu。這樣定義的優點是可簡化複雜的公式書寫,煩瑣的畫圖指令, 故適合將重複性的一堆指令,化繁為簡。哇$!!!$學到重點了,其他就剩舉一反三了;注意$!$
這是無參數的用法,有參數的用法稍後說明。
\begin{Verbatim}[frame=single,firstline=1,label=Theorem]
\begin{thm} 三角形三內角和為$180^\circ$。
\end{thm}
\begin{pf}
因為$\ldots$所以$\ldots$。餘類推。
\end{pf} 
\end{Verbatim}
產生\index{\LaTeX!\textbackslash index}
\begin{thm} 三角形三內角和為$180^\circ$。  
\end{thm}
及其證明
\begin{pf}
因為$\ldots$所以$\ldots$。餘類推。
\end{pf} 
再舉一例
\begin{Verbatim}[frame=single,firstline=1,label=Problem]
\begin{pr}
作業內容在此。
\end{pr}
\end{Verbatim}
產生\footnote{如何定義解答環境? Ans: \textbackslash newtheorem\{ans\}\{解答\}[chapter]}
\begin{pr}
作業內容在此。
\end{pr}

有些學科,例如電腦資訊科學,需將演算法{\tt(Algorithm)}寫出,這時也可用定義新理論的方式來達成。舉例如下
\index{演算法}。
\begin{Verbatim}[frame=single,firstline=1,label={An algorithm}]
\newtheorem{algorithm}{演算法}[chapter]
\begin{algorithm}[An Algorithm]
\hfill\par           % 保持 algorithm 與 tabbing 距離
\begin{tabbing}
1. \hspace{1cm} \=For $k=1$ to $k^{\max}$\\  % iterations
2. \> For \hspace{0.5cm} \=$i=1$ to $n$\\    % iterations
\>\>Set
\[
x_i^{(k)} =
\frac{b_i-\sum_{j=1}^{i-1}a_{ij}x_j^{(k)}
-\sum_{j=i+1}^{n}a_{ij}x_j^{(k-1)}}%
{a_{ii}}
\]
\\
3. \> \textrm{If} $\|\vec{x}^{(k)}-\vec{x}^{(k-1)}\| < 
\epsilon$, \textrm{stop.}
\end{tabbing}
\end{algorithm}
\end{Verbatim}
其中\verb+\=+是第一、二行的定位點,分別設為向右1及0.5公分。\verb+\>+則為第二、三行以後,各行的對齊點點。其結果為 \index{定位點}\index{對齊點}
\begin{algorithm}[An Algorithm]
\hfill\par   % 防止 algorithm 與 tabbing 環境間結合
\begin{tabbing}
1. \hspace{1cm} \=For $k=1$ to $k^{\max}$ \\   
2. \> For \hspace{0.5cm}\=$i=1$ to $n$\\   
\>\> Set
$
x_i^{(k)} =
\frac{b_i-\sum_{j=1}^{i-1}a_{ij}x_j^{(k)}
-\sum_{j=i+1}^{n}a_{ij}x_j^{(k-1)}} {a_{ii}}
$\\
3. \>\textrm{If} $\|\vec{x}^{(k)}-\vec{x}^{(k-1)}\| < \epsilon$, \textrm{stop.}
\end{tabbing}
\end{algorithm}
%其中第一段英文字變成{\it italic}斜體字是因{\tt tabbing}環境造成,此時在宣告區加入\verb+\usepackage{amsthm}+可改為正楷體。或如第三段英文所示,用\verb+\textrm{text}+將文字段落包住。
\index{\LaTeX\ 環境!tabbing}
\index{\LaTeX\ 環境!tabbing!\textbackslash =}
\index{\LaTeX\ 環境!tabbing!\textbackslash $>$}
\subsection{表格}
表格變化較多。最基礎的表格則用 {\tt tabular} 及 {\tt table} 環境。請注意!
{\tt tabular}與{\tt array}有點相似,前者是文字環境(可有數學\$\$),後者是數學環境(可有文字\textbackslash mbox)。\\
\begin{Verbatim}[frame=single,firstline=1,label=Tabular 1]
\begin{center}
\begin{tabular}{ccc}
實驗 & 方法 1 & 方法 2 \\  
\hline \hline
1 & 1275.6 & 5.38309 \\ 
2 & 2345.3 & 3.48736
\end{tabular}
\end{center}
\end{Verbatim}
會產生\footnote{除center(置中)環境外,還有flushright(置右),flushleft(置左)。}
\index{\LaTeX!\textbackslash hline}
\begin{center}
\begin{tabular}{ccc}
實驗 & 方法 1 & 方法 2 \\  \hline \hline 
1 & 1275.6 & 5.38309  \\ 
2 & 2345.3 & 3.48736
\end{tabular}
\end{center}  
\index{\LaTeX\ 環境!center}%
或這樣寫有邊框且會自動放置於頁面之上中下,因為最外層是{\tt table}環境。它與{\tt figure}環境都是浮動環境(\tt floating environments),顧名思義就是讓\LaTeX\ 決定位置{\tt hbt}。\\ 
\begin{Verbatim}[frame=single,firstline=1,label=Tabular 2]
\begin{table}[!hbt]
\centering
\begin{tabular}{|c|c|c|} \hline 
 A & B & C \\ \hline \hline
 D & E & F \\
 1 & 2 & 3 \\ \hline
\end{tabular}
\caption{實驗結果} 
\label{bookstruc1}
\end{table}
\end{Verbatim}
\index{\LaTeX!\textbackslash centering}
\index{\LaTeX!\textbackslash hline}
\begin{table}[!hbt]
\centering  
\begin{tabular}{|c|c|c|} \hline        
A & B & C \\ \hline \hline
D & E & F \\
1 & 2 & 3 \\ \hline 
\end{tabular}
\caption{實驗結果}
\end{table} 
\index{\LaTeX\ 環境!flushleft}
\label{bookstruc1} \index{\LaTeX\ 環境!tabular}
\index{\LaTeX\ 環境!flushright}
其中表格可能會出現在{\tt top}(上方),{\tt bottom}(下方),{\tt here}(在此),驚嘆號$!$ 表示則由\LaTeX排版決定{\tt hbt}擇一。這裡顯示的是最簡單的表格,較複雜的表格分隔亦多,主要是活用指令\verb|\multicolumn{欄數}{對齊}{名稱}|。這指令使用後相當於佔據相等的\{欄數\}。做表格時請一列一列思考,需水平線則用\verb|\hline或\cline|,需垂直線則用垂直線\verb+|+。

\begin{Verbatim}[frame=single,firstline=1,label=A bit complicated tabular]
\begin{table}[!hbt]
\centering
\begin{tabular}{cc||cccc}                \hline
\multicolumn{2}{c}{matrix} &        %兩欄 置中 名稱 定位
\multicolumn{4}{c}{$\bar k=k-e_i-e_j$}\\%四欄 置中 數學式
\cline{1-2} \cline{3-6}
$ij$ & $(e_i+e_j)$ & 30 & 21 & 12 & 03\\ \hline \hline
11   & 20          & 10 & 01 & -  & - \\ 
12   & 11          & -  & 10 & 01 & - \\  
21   & 11          & -  & 10 & 01 & - \\  
22   & 02          & -  & -  & 10 & 01    
\end{tabular}
\end{Verbatim}
可做出表格如下。
\begin{table}[!hbt]
\centering
\begin{tabular}{cc||cccc}                \hline
\multicolumn{2}{c}{matrix} & 
\multicolumn{4}{c}{$\bar k=k-e_i-e_j$}\\ 
\cline{1-2} \cline{3-6}
$ij$ & $(e_i+e_j)$ & 30 & 21 & 12 & 03\\ \hline \hline
11   & 20          & 10 & 01 & -  & - \\ 
12   & 11          & -  & 10 & 01 & - \\  
21   & 11          & -  & 10 & 01 & - \\  
22   & 02          & -  & -  & 10 & 01  
\end{tabular}
\end{table}
\index{\LaTeX\ 環境!tabular!\textbackslash multicolumn}
更複雜的表格是可行的,只是需上網自學,然而上述初步的理解足以讓你上網再深入學習,不難了,都是舉一反三的應用了。



\subsection{小頁}
小頁又稱為{\tt minipage},有時候整頁模式須用兩欄式表現對照,這時可這樣做\\
\begin{Verbatim}[frame=single, firstline=1,label=Minipage]
\begin{minipage}[b]{0.49\textwidth}
{\bf A {\rm two-\emph {column} setup} demon.}
\end{minipage}
\rule{2pt}{2cm} % 中垂線 可移除
\begin{minipage}[b]{0.49\textwidth}
這是兩欄式的寫法,3種字體。
\end{minipage}
\end{Verbatim}
可產生如下的結果\footnote{三欄式如何設計? Ans:3個連續的{\tt \{minipage\}\{0.3\textbackslash textwidth\}}環境。}\\
\begin{minipage}[b]{0.49\textwidth} \index{\LaTeX!\textbackslash textwidth}
{\bf A {\rm two-\emph {column} setup} demon.}  \index{\LaTeX!\textbackslash bf}\index{\LaTeX!\textbackslash emph}
\end{minipage}
\rule{2pt}{2cm}%
\begin{minipage}[b]{0.49\textwidth}
這是兩欄式的寫法,3種字體。
\end{minipage}\\ \index{\LaTeX!\textbackslash footnote}

兩欄式\footnote{真正的兩欄式亦可在第一行{\tt documentclass}中加入{\tt [twocolumn]}。但此論文不適合。}的內容某一欄可以是圖、表、或方程式,另一欄可以是解釋或對照的圖表。完全看設計者的想法。

\subsection{列舉} 
寫論文時亦需要條列式的列舉,將重點列出。最常見的用法有兩種,都是以\verb+\item+為主。第一種用法{\tt enumerate}會自動編號。適合有先後次序的概念,例如實驗步驟。\index{\LaTeX\ 環境!enumerate}\\
\begin{Verbatim}[frame=single,firstline=1,label=Enumerate]
\begin{enumerate}
\item 這是重點一
\item 這是重點二
\item 其他\ldots
\end{enumerate}
\end{Verbatim}
會產生
\begin{enumerate}
\item 這是重點一
\item 這是重點二
\item 其他\ldots
\end{enumerate}

第二種用法{\tt itemize}不會自動編號,適合無先後次序的重點陳述,但保留彈性使用(包括自行編號)。
\index{\LaTeX\ 環境!itemize}\\
\begin{Verbatim}[frame=single,firstline=1,label=Itemize]
\begin{itemize}
\item[甲] 使用文字做為標記
\item[!] 使用符號做為標記
\item    沒用也可以,但預設是圓點。
\end{itemize}
\end{Verbatim}
[\ldots]內為使用者自訂,可以是文字縮寫,符號,或其他有意義的設計。這樣寫會產生
\begin{itemize}
\item[甲] 使用文字做為標記
\item[!] 使用符號做為標記
\item    沒用也可以,但預設是圓點。
\end{itemize}
甚至兩種環境混合使用\\
\begin{Verbatim}[frame=single,firstline=1,label=Enumerate+Itemize]
\begin{itemize}
\item[甲] 使用文字做為標記
\begin{enumerate}
\item 它會自動內縮
\item 故一目了然
\end{enumerate}
\item[!] 使用符號做為標記
\item    沒用也可以,但預設是圓點。
\begin{enumerate}
\item 這種巢狀式的結構
\item 在程式語言常見
\end{enumerate}
\end{itemize}
\end{Verbatim}
表現出多樣性的列舉環境
\begin{itemize}
\item[甲] 使用文字做為標記
\begin{enumerate}
\item 它會自動內縮
\item 故一目了然
\end{enumerate}
\item[!] 使用符號做為標記
\item    沒用也可以,但預設是圓點。
\begin{enumerate}
\item 這種巢狀式的結構
\item 在程式語言常見
\end{enumerate}
\end{itemize}

\subsection{插圖}
實驗室的結果時常以圖形表示,而往往這些圖未必以\LaTeX\ 做成,要如何將此類圖檔加入論文內? 是有方法的。 這樣寫將可將事先做好的圖檔({\tt png,jpg,jpeg,pdf})加入\LaTeX論文內。 
\index{\LaTeX!\textbackslash includegraphics!png}\index{\LaTeX!\textbackslash includegraphics!jpg}\index{\LaTeX!\textbackslash includegraphics!pdf}
\index{\LaTeX!\textbackslash includegraphics!jpeg}

\LaTeX\ 皆受不同圖檔格式{\tt png,jpg,jpeg,pdf,mps}可加入文稿內。若檔名相同則\LaTeX\ 會依下列優先次序讀取
{\tt png>pdf>jpg>mps>jpeg}。\\
\begin{Verbatim}[frame=single,firstline=1,label=Every figure]
\begin{figure}[!hbt]
\centering
\includegrapics[width=x,height=y,scale=z]{foo.pdf}  
\caption{曲線圖1}
\label{Fig1}
\end{figure}
\end{Verbatim}
其中 {\tt[!hbt]}的意義同前。{\tt x, y}必須寫入單位 cm(公分),in(英吋),放大或縮小 $0 \le {\tt z} \le 1$。
這樣寫會產生校徽。
\begin{figure}[!hbt]
\centering
\includegraphics[scale=3]{NCUlogo} 
\caption{中大校徽放大三倍}
\label{Fig1}
\end{figure}%
顯然地,圖可能不只一張,若每張都要這樣輸入,就不方便了。\LaTeX{}有想到這點,我們可定義新指令({\tt \textbackslash newcommand})如下\\
\begin{Verbatim}[frame=single,firstline=1,label={Macro with parameters}]
\newcommand{\insertfig}[2]{
\begin{figure}[!hbt]
\centering
\includegraphics[scale=2]{#1}
\caption{#2}
\label{Fig:#1}
\end{figure}
}
\end{Verbatim}
其中[2]代表有兩輸入變數。\#1=圖檔名,\#2=該圖的標題,且自動以檔名做該圖的標記,可適當時引述({\tt \textbackslash ref(Fig:\#1)})\footnote{\textbackslash cite是文獻的引述;\textbackslash ref 是圖,表,數學式的引述。不要混淆了。}。這定義({\tt macro})可寫在任何地方,但多數是寫在宣告區({\tt preamble})\textbackslash document前,如果很多這樣的定義,則放在{\tt mypreamble.tex}內,再放入宣告區,以保持整潔。使用時只要在適當位置寫
\index{\LaTeX!\textbackslash newcommand}
\begin{Verbatim}[frame=single,firstline=1,label={Macro usage}]
\insertfig{NCUlogo}{中央大學校徽放大二倍}
\end{Verbatim}
則會產生(指令少很多了)。
\insertfig{NCUlogo}{中央大學校徽放大二倍} 
這些圖表\ref{Fig:NCUlogo}所在位置的相關頁碼都會自動加入圖目錄({\tt LoF})及表目錄({\tt LoT})中。同理,章節的頁碼亦自動加入目錄中。作者不必擔心。
\index{\LaTeX!LoF}\index{\LaTeX!LoT}\index{\LaTeX!ToC}


在此介紹其他好用的\LaTeX\ 指令用於排版:\index{\LaTeX!\textbackslash newline}
\begin{itemize}
\item \textbackslash cleardoublepage:從奇數頁開始。 \index{\LaTeX!\textbackslash cleardoublepage}
\item \textbackslash clearpage:從下一頁開始。       \index{\LaTeX!\textbackslash clearpage}
\item \textbackslash linebreak,\textbackslash newline,\textbackslash\textbackslash:強置換行。  
\index{\LaTeX!\textbackslash linebreak}
\item \textbackslash noindent:第一行不內縮。        \index{\LaTeX!\textbackslash noident}
\item \textbackslash index\{level one!level two!level three\}:索引指令\footnote{想想看,寫出含3個變數的{\tt index macro}。}。
\item \{\textbackslash bf text\}:黑體強調{\bf text},其他 \textbackslash it, \textbackslash tt 等雷同。
\end{itemize}
以上\LaTeX\ 介紹並非全部,只是常見的基礎,都頗直覺式/口語式的寫法(專業術語有一個漂亮的名子--文學編程{\tt literate programming}),可知並不難學\footnote{LyX有提供結構性的寫法,更簡化入門門檻,請用關鍵字收尋。}。開始上網學習了,用 {\tt Google} 搜尋 {\tt latex basics}或{\tt latex beginner} 關鍵字,一小時後你就比現在深入多了。可以用\LaTeX{}寫報告或論文了。

\begin{table}[hbt!]
\caption{學習方向}
\begin{enumerate}
\index{\LaTeX\ 環境!enumerate}
\item{\color{blue} 初步:目標是抓到概念。}\\
\url{http://www.cs.nthu.edu.tw/~cherung/teaching/2009cs5321/link/latex.pdf}\\
\url{http://www.iu.hio.no/~frodes/rm/ppt}
\item{\color{cyan} 入門:目標是了解、熟習語法的規律性。\\ 
\url{http://spe.num.edu.mn/altankhuu/lesson/comp101/nemelt1/latex_for_beginner.pdf}}
\item{\color{yellow}
中級:要下載些不錯的文件,以供隨時查閱。}\\ 
\url{http://people.debian.org.tw/~koster/latex/lshort-zh-tw.pdf}\\
\url{http://www.tug.org.in/tutorials.html}
\item
高級:{\tt Notable books 
\begin{itemize}
\item The \LaTeX{ }Companion, 2nd edition, by  Frank Mittelbach {\it et al}, and many others in NCU Library.
\item TeX for the impatients, by Paul W. Abrahams, Kathryn A. Hargreaves, Karl Berry. (Free, can be found in the Internet, CTAN)
\item TeX by topics, by VICTOR EIJKHOUT (Free, can be found in the Internet)
\item [原著] \cite{knu84,lam94}
\end{itemize}}
\end{enumerate}
\label{res}
\end{table}%
請問此表\ref{res}是用哪些環境指令畫出的?\footnote{\tt table,caption及enumerate。}
完全不懂\LaTeX{}者,至少走完入門階段,再考慮是否用此論文套件寫你的論文。已經懂\LaTeX{}的研究生應會選擇此套件。
\index{\LaTeX\ 環境!tabular}
\index{\LaTeX\ 環境!table}
\index{\LaTeX\ 環境!figure}

\vfil  \index{\LaTeX!\textbackslash vfil}

\chapter{文獻製作}\index{\LaTeX!\textbackslash chapter}
依此類推,同理可用,本章將說明論文的檔案夾"{\tt NCU}論文"內含那些的文件,及簡單說明如何製作參考文獻。並說明現存問題及將來發展。檔案下載後請改為"某某某論文"。
\section{如此這般}
這樣繼續打字,製表,作圖,就可完成論文撰寫。
\subsection{檔案}
此論文範例放於"{\tt NCU}論文"資料夾,內含下列檔案。
\fvset{frame=single,firstline=1,lastline=24}
\VerbatimInput{readme.txt}  \index{\LaTeX!\textbackslash VerbatimInput}
{\color{red} \Huge \tt Don't delete *.tex, *.cls, *.bib.} 
\section{引用致謝}
研究成果總有參考文獻,畢竟我們都是站在巨人的肩膀上再向前創新發展。引述別人的成果可表示我們對他人的感謝。\LaTeX\ 提供兩種方式達到此效果。(甲)
若參考文獻不多者(少於十篇),可照此檔案夾內{\tt bibli.tex} 用\verb+\bibitem+的寫法打入相關資訊, %但要注意不要刪除錯誤的符號。(乙) 若文獻多時,則建議先建檔,譬如檔名為{\tt myfoo.bib}, %其主要結構如下,更多其他結構可參考網路資訊。
% \index{Bibliography!\char64 article}
% \index{Bibliography!\char64 inproceedings}
% \index{Bibliography!\char64 book}
% \index{Bibliography!\char64 unpublished}

\begin{Verbatim}[frame=single,firstline=1,lastline=32,rulecolor=\color{red},label=Typing up myfoo.tex]
@article{paper,
title      = "Title",
author     = "Author A and Author B",
journal    = "Name of journal",
volume     = "6",
number     = "2"
pages      = "xxxx--xxxx",
month      = feb,          % 不用引號
year       = "2012"
}
@inproceedings{conference,
author     = "First author and Second author",
title      = "Title of the conference paper",
booktitle  = "Proceedings of the $X^{th}$ Conference on XYZ",
year       = "2006",
pages      = "xxx--xxx",
volume     = "3",
month      = oct           % 不用引號
}
@book{ethinking,
author     = "Jesse LO",
title      = "eThinking in Circuits with PSpice",
year       = "2012",
month      = sep,
note       = "ISBN 978-957-41-8721-8"
}
@unpublished{ncuthesis,
author     = "Jesse LO",
title      = "碩博士論文(Xe)\LaTeX使用手冊",
month      = "11/30",          
year       = "2011"
}
\end{Verbatim}
%如果你因這檔案而學到\LaTeX{}且獲益良多,讓你在極短的時間內,即時、快速完成漂亮的論文,你可以考慮給它一個"讚",將這檔%案引述於您的論文參考文獻中。讓更多的中央大學碩士、博士生了解\LaTeX\ %、或產生興趣,進而用於論文撰寫或出版書籍。\ldots \ldots 文獻建立完成後、存檔成{\tt %myfoo.bib},別忘了副檔名是{\color{red}\tt .bib},不是{\color{red}\tt .tex}。然後於主檔{\tt %masterthesis.tex}內最後幾行中刪除\verb+include{bibli}+,再加入兩行\\
\begin{Verbatim}[frame=single,rulecolor=\color{red},label=Add this]
\bibliographystyle{style}
\bibliography{myfoo}
\end{Verbatim}
其中{\tt style}有四種選擇\\ 
{\tt plain} -- 照英文字母排序\\
{\tt alpha} -- 照{\tt plain}但[1,2,3,4]用英文名({\tt given name})及年份排序\\
{\tt abbrv} -- 照{\tt plain}但以英文姓({\tt last name})及年分排序\\
{\tt unsrt} -- 照論文中引述先後順序排序
這樣就加入主檔了,引述時,在論文適當處這樣寫\verb+\cite{paper,conference,ncuthesis,ethinking}+
會產生文獻列印於後。一切引述應出現的地方,編號的安排,\LaTeX都會負責。
因為這四個風格都是最陽春的,建議採用從網路下載{\tt IEEEtran.sty}或個人喜歡的風格,參考文獻會漂亮許多。


\section{安裝引擎}
\subsection{Window}
一小時過去了,執行{\tt masterthesisXe.tex}試試看,還是不行。為什麼?因為還未安裝啦。搜尋{\tt MiKTeX}可得網頁\url{http://miktex.org/2.9/setup}在左邊有一{\tt MiKTeX Portable}的英文字,按一下,開始照說明安裝。初次執行時{\tt MiKTeX}會自動要求下載巨集更新,請按{\tt Yes},不一會兒,你就有一份隨身{\tt USB}的\LaTeX{}隨身攜帶,隨時可玩。{\tt MiKTeX}檔案夾已內含編輯器 {\tt TeXworks}。需執行{\tt miktex-portable.cmd}

\subsection{Android}
目前平板電腦,智慧型手機採用{\tt Android}系統者,皆可至{\tt Play}商店下載免費\LaTeX\ 引擎{\tt \TeX portal}。亦是隨身攜帶的\LaTeX\ 不需網路連接(只有安裝時需網路)。


\section{現在未來}
此體裁檔雖通過機械系六本70--80頁左右的碩士論文編譯及{\tt Window}環境測試,相信仍有改進空間。 回報錯誤或有更簡潔的\LaTeX\ /Xe\LaTeX{}寫法,請通知{\tt jclo\char64 cc.ncu.edu.tw},將盡速了解、更正及誌謝,但非所有提問或要求皆處理,謝謝。

\subsection{已知問題}
\begin{enumerate}
\item   \textbackslash marginpar (這指令能在左右空白處加註解),在Xe\LaTeX\ 編譯後若有寫,應出現偶數頁左註解或奇數頁右邊註解,有時卻不出現。
\item 因外來巨集({\tt macro})是由各愛好者所寫,更新時可會造成相衝,而有編譯問題,建議直接按{\tt enter}鍵 繼續編譯,或找到該問題行並將該行註解(\%)。
\index{\LaTeX!\textbackslash raggedleft}
\index{\LaTeX!\textbackslash raggedright}
\item {\tt calculator}巨集應該會自動下載,若沒有則需手動安裝。目前資料夾內有該檔案。
\end{enumerate}

\subsection{未來方向}\index{\LaTeX!\textbackslash subsection}\index{\LaTeX!\textbackslash chapter}
\begin{itemize}
\item 浮水印。
\item 在不同系統上測試。
\item {\tt LyX (要將lyx轉成tex檔)}  \index{\TeX!}
\end{itemize}
\vfill
\section{歷史更新}
\begin{tabular}{l@{:}l}
Ver 1.04 & 2013/06/30\\
& 加入{\tt Android} \LaTeX\ 資訊。\\
&修正目錄連結不正確,少了\verb|\cleardoublepage|。\\ 
&新增共同指導教授欄位。\\
&三本論文Window測試成功。\\
Ver 1.03 & 2013/01/24\\
&200+次下載。\\
&加入演算法環境。\\
&{\tt MiKTeX}更新後,{\tt RequiredPackage\{xltxtra\}}\\
&及論文子標題(\tt subtitle)內\verb|\XeLaTeX|\\
&會造成編譯錯誤。故除去後即可編譯。\\
Ver 1.02 & 2012/11/30\\
&{\tt ToC}對齊。\\
&加入{\tt ncuthesisXe.cls}檔。 \\
&加入文獻製作。\\
&\textbackslash {\tt today} 中文化。\\
&加入Xe\LaTeX{}編譯。\\
&可顯示文字外框,未完稿功能。\\
&更多數學例題。\\
&加入學習\LaTeX{}資訊。\\
&{\tt TeXLive2009/Ubuntu 12.04}。\\
&插頁頁碼。\\
&每段內縮。\\
&目錄超連結。\\
&教務處測試成功。\\
&新增\textbackslash bookbone,\textbackslash printpapersize 指令。\\
Ver 1.01 & 2012/05/30 \\ 
&根據教務處範例製作({\tt form-03-02-2.doc})。\\
&\url{http://pdc.adm.ncu.edu.tw/Register/}\\
&三本論文測試成功。
\end{tabular}\\
這{\tt ncuthesis}使用說明書是以\fmtname, 版本~\fmtversion 製作。
